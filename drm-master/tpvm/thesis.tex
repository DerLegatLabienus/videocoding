\documentclass[conference]{IEEEtran}
\usepackage{cite}
\usepackage{hyperref}
\begin{document}
\title{Truly Protected Digital Video}
\author{\IEEEauthorblockN{Asaf David}
\IEEEauthorblockA{School of Computer Science\\
    Tel Aviv-Yaffo Academic College\\
    Tel Aviv, Israel\\
Email: asafdavi@mail.mta.ac.il}
}
\maketitle

\begin{abstract} 

With the Internet becoming the leading platform for multimedia content
distribution, the proper licensing of such data is turning to be a major
concern for media owners and providers. DRM solutions are the standard tool used
to enforce proper terms of content use. While many DRM methods were proposed
and implemented over the years, ways to circumvent them are usually
found short time upon release, often by exploiting the ability of a
malicious client to intercept the tool in runtime and obtain the
encryption key. This research proposal suggests a new DRM solution,
specifically aimed at streaming video, which tries to prevent this
vulnerability; The decryption of the video and the rendering of frames
happen atomically inside a virtual machine, which establishes a controlled
and trusted environment. The machine-code instructions run by the VM
are encrypted per client. The decryption of instructions happen only at
runtime, using a key which is held securely inside the CPU, inaccessible
to any other process. Moreover, the genuinity of the system is verified
prior to the key exchange. We believe this system achieves an improved
level of security compared to existing DRM solutions. The project's
adviser will be Dr. Nezer Jacob Zaidenberg.

\end{abstract} 
\IEEEpeerreviewmaketitle

\section{Introduction} 

The recent years have shown a rapid increase in multimedia usage in
various applications. The technological advances in terms of consumer
bandwidth paved the way for the Internet to become the leading platform
for distribution of such content. These days, more and more consumers
prefer to download the video content (either as streaming media or
container files) rather than buy it in stores as they used to.

One of the major concerns introduced by this trend is the proper
enforcement of content licensing. These days, media providers make
tremendous efforts to prevent third parties from redistributing licensed
content. It has been shown that video piracy may lead to major financial
losses; For example, a study held in the Institute for Policy Innovation
in attempt to accurately assess the U.S industry losses to piracy has
estimated the total annual loss in \$20.5 billion \cite{1}.

In attempt to prevent piracy, \emph{Digital rights management} (DRM)
solutions are utilized. In essence, a DRM solution is a software tool
meant to enforce a policy of digital content usage after sale. We
can track such tools back to the sixties, with new methods appearing
continuously to this day. While publicly detailed information regarding
the inner working of such tools, and especially their security features,
is usually non-existent (due to obvious reasons), the basic structure
is usually generic. It consists of at least three components: a content
packager, a license server and a client DRM. The main difference between
the various DRM solutions is the supported usage models. Among the best
known DRM solutions are Microsoft DRM (bundled with Microsoft Media
Player), Apple FairPlay (used for example in the iTunes store) and
Adobe Flash Access. All of these have been hacked before. Note that as
opposed to CableTV security, which usually contains trust model based on
tamper-resistant hardware inside a Set-Top Box, A DRM solution have to
rely on generic computers and on software tamper-resistance.

\section{Requirements}

The main requirement of our system is the prevention of unlicensed users
from viewing the copyrighted video material, while still preserving
smooth and enjoyable watching experience for legitimate clients. We
will enforce the restriction of unauthorized viewing using cryptographic content
encryption; the video will be encrypted by the content packager and
decrypted by the client viewer right before being played. Using the
client's public key for encryption will guarantee that the client is the
only one who can decrypt the content.

It's generally impossible to prevent a malicious user from capturing the
decoded content (that is, the actual frames rendered to the screen, for
example by recording the HDMI output of the computer or placing video
camera in front of the monitor) in attempt to redistribute it. As it
such actions usually lead to considerable reduction of video quality,
we further ignore this threat. Instead, we consider the video in in its
encoded by decrypted form as the main asset we want to protect. As such,
it must be accessible solely by the viewer process. Thus the viewer must
perform both the decryption and decoding stages atomically. Moreover,
the video decryption key must be held securely, out of reach by either
users or any other processes running alongside the viewer. This problem
was tackled before \cite{2} and we plan to use a similar solution, as
outlined in the next section.

\subsection{Selective encryption}

While encrypting to entire video bit-stream (also known as the
naive encryption model) can guarantee complete security, it is
usually infeasible when targeting real-time scenarios due to the high
computational costs imposed on the encoder and decoder. \emph{Selective
encryption} methods aim to solve this problem by only encrypting a
subset of the video bit-stream. Many selective encryption methods have been
proposed before, trading off various measures such as percentage of
encrypted content, complexity, visual degradation and error tolerance
(for example, see \cite{3} for a comparison of many selective
encryption methods, specifically targeting H.264 coded video). As our
tool mainly targets the commercial market, we're more interested in
prevention of unlicensed usage rather than total video confidentiality.
(In other words, selective encryption schemes which which may leak some image details but still 
seriously degrade the viewing quality are accepted). This will reduce the required
computational complexity of the encoder/decoder and will allow to use
the tool in real-time scenarios. The exact selective encryption method
will be chosen in a later stage.

\subsection{Management capabilities}

The system will also allow basic management capabilities - it would be
possible to allow/prevent some user from watching videos. It would also
be possible to display usage statistics - for example, how many times
was each movie watched. This means that the server must be notified by
the Viewer whenever a user watches a movie.

\section{System Design} 
Our system contains two separate sub-systems: 
\begin{enumerate} 
    \item \textbf{Distributor} 
        This sub-system is used by the video distributor. It handles the
        encoding and encrypting of the video and its distribution to the end
        user. It also contains a data-base for storing information about the
        system users and stored movies, and to hold usage statistics.
    \item \textbf{Viewer} 
        This sub-system is used by the end user. It is the secure
        environment in which the video is decoded and played. The viewer
        is specifically compiled for each user according to the user
        key, as explained below.
\end{enumerate}

The Viewer is essentially a video player wrapped in a specially crafted
virtual machine. This VM establishes the secure and trusted environment
required to protect the content - It can only run machine-code in
an encrypted form, which was specifically compiled to target the
client running the Viewer. Since decryption is a relatively expensive
operation, we can't have encrypt all instructions. Instead, we use
the notion of \emph{scrambled} instructions. A scrambled instruction
has various parts of it permuted (for example - the registers or the opcode)
so effectively it cannot be run with the correct permutation translation table.
This translation table is repeatedly updated by inserting a special
instruction which defines a new translation table once every few regular
instructions. Only these special instructions are encrypted, so the
decryption time is reduced greatly. Once a translation table is known,
the descrambling of the regular instructions is very fast.

Obviously, the key used to decrypt the special instructions must not be
accessible to other processes, and measures must be taken to achieve
this. One idea is to store the generated key in the CPU cache and
immediately lock this cache location from further writes. Once a client
generated this key it must transfer it security (using a key-exchange
algorithm) to the server, where the specific Viewer for the client will
be built.

In addition to the code-decryption key which was defined above, the
Viewer also generates and holds in a similar fashion a video-decryption
key. This key is used to decrypt the selective encrypted video
bit-stream. The trusted environment will be used to prevent other
processes from acquiring the video decryption key. The security of
the system depends on the fact that the keys were generated in a real
(non-virtual) environment, where reading the CPU registers is considered
impossible. Hence the host must prove itself as such to the server. A
verification algorithm that verifies this is proposed in \cite{4} and
will be used in our system as well.

The video encoder/decoder must handle the selective
encryption/decryption, respectively. We will use H.264, a
state-of-the-are video compression system, as the video standard for our
system. H.264 is currently the most widely-deployed video compression
system and has gained a dominance comparable only to JPEG for image
compression due to its excellent coding efficiency. Various open-source,
production ready H.264 encoders and decoders exist (such as x264\cite{5}
and FFMpeg\cite{6}) and could be used as a basis for modification.

\section{Timeline} 
We plan to have an operational version ready by September 2013 according to the following plan:
\begin{enumerate} 
    \item Reading literature and selecting encryption algorithm by February, 2013. 
    \item Implementing the selective encryption algorithm in the chosen codec by March, 2013.
    \item Wrapping the decoder in the VM secure environment by May, 2013
    \item Building the server side platform wrapping the encoder by June, 2013
    \item System integration, tests and documentation by August, 2013.
    \item Project final paper preparation by September, 2013. 
\end{enumerate}

\bibliographystyle{IEEEtran} 
\begin{thebibliography}{99} 
    \bibitem{1} S.E. Siwek. \href{www.copyrightalliance.org/files/u227/CostOfPiracy.pdf}{The True Cost of Motion Picture Piracy to the U.S. Economy}. In {\it Policy Report 186}, Sept. 2006
    \bibitem{2} N. Zaidenberg, M. Kiperberg, A. Averbuch. \href{www.cs.tau.ac.il/~kiperber/truly.pdf}{An Efficient VM–Based Software Protection}. In {\it Network and System Security (NSS), 2011 5th International Conference on}, Sept. 6, 2011. 
    \bibitem{3} T. Stuetz, A. Uhl. \href{}{A Survey of H.264 AVC/SVC Encryption} in {\it Technical Report 2010-10}, Dec. 2010
    \bibitem{4} R. Kennel, H. Jamieson. \href{http://static.usenix.org/events/sec03/tech/kennell/kennell.pdf}{Establishing the Genuinity of Remote Computer Systems}. In {\it Proceedings of the 12th USENIX Security Symposium}, 2003.
    \bibitem{5} x264. \url{http://www.videolan.org/developers/x264.html}
    \bibitem{6} FFMpeg. \url{http://www.ffmpeg.org}

\end{thebibliography} 
\end{document}
